\documentclass{article}
\usepackage{xeCJK}
\usepackage{graphicx}                % 嵌入png图像
\usepackage{color,xcolor}            % 支持彩色文本、底色、文本框等
\setCJKmainfont{Microsoft YaHei} 
\XeTeXlinebreaklocale "zh"
\title{ \textbf{ MISTAKE}}
\author{ 李顶龙}
\date{}
\begin{document}

\maketitle
\section{ cdq分治}
\begin{itemize}
	\item 不要排结构体,因为排结构体到时候还要排回来。
	\item 线段树打时间戳不要memsize().
	\item 第二维排序的时候一定要双关键字排序
\end{itemize}

\section{ 朱刘算法}

\begin{itemize}

\item vis[i] = i 
\item 最后更新边的时候注意顺序
\item 根节点的处理要很小心
\item root缩点之后标号也要变化

\end{itemize}

\section{二分图}
\begin{itemize}

\item DAG最小路径覆盖(最小的不相交路径覆盖所有顶)=|V(G)|-最多的不相邻的边:拆点之后u->v的边变为Xu->Yv,再求最大匹配即可(不相交的边尽量多)
\item 最大匹配=最小点覆盖;
\item 独立数=点数-匹配数=最小边覆盖
\item 求有向无环图最小可相交路径,可以先求floyed求传递闭包,转化成最小不可相交路径
\end{itemize}

\section{ 倍增}
\begin{itemize}

\item 忘了先化到同一深度
\item 最后判断u,v不相等
if (u != v)
	ans = max(ans, max(big[u][0], big[v][0]));
\item 倍增数组里边的下标一定要注意
\end{itemize}

\end{document}



